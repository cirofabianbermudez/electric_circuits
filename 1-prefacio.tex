	\chapter*{Prefacio}
		\addcontentsline{toc}{chapter}{Prefacio}
		
		Uno se preguntará por qué se selecciono la foto del NASCAR para la portada. En realidad, se seleccionó por varias razones. Obviamente, es muy excitante, ya que se trató de que McGraw-Hill modificara el auto que va a la delantera con el logo de la compañía pegado sobre el y a “Alexander y Sadiku” ¡al otro lado del auto! Otra razón, no tan obvia, es que la mitad del costo de un auto nuevo lo representa su electrónica (¡circuitos!). Sin embargo, la razón más importante es que un ¡auto ganador necesita de un “equipo” para lograrlo! Y trabajar juntos como equipo es muy importante para el ingeniero exitoso y algo que se fomenta ampliamente en este texto.
		
\par\noindent\rule{\textwidth}{0.4pt}

\noindent
\textbf{\uppercase{Caracteristicas}}

\noindent
\textbf{Conservadas de ediciones anteriores}

Los objetivos principales de la tercera edición de este libro se mantienen iguales con respecto a la primera y segunda ediciones, a fin de presentar el análisis de circuitos de una manera que sea más clara, más interesante, y más fácil de comprender que en otros textos, y para ayudar al estudiante a que comience a ver la “diversión” de la ingeniería. Este objetivo se logra de las formas siguientes:

\begin{itemize}
	\item \textbf{Introducción y resumen en cada capítulo} 
	
	Cada capítulo inicia con un análisis acerca de cómo desarrollar las habilidades que contribuyan al éxito en la solución de problemas así como al éxito en la profesión o con una plática orientada a la profesión sobre alguna subdisciplina de la ingeniería  eléctrica. A esto lo sigue una introducción que vincula el capítulo con los capítulos anteriores y plantea los objetivos de dicho capítulo. Éste finaliza con un resumen de los puntos y fórmulas principales.
	
	\item \textbf{Metodología en la solución de problemas}
	
El capítulo 1 presenta un método de seis pasos para resolver problemas
sobre circuitos, el cual se utiliza de manera consistente a lo largo del texto y de los suplementos multimedia a fin de promover las prácticas más actuales para la solución de problemas.

	\item \textbf{Estilo de la escritura amigable para el estudiante}
	
Todos los principios se presentan de una manera clara, lógica y detallada. Tratamos de evitar redundancias y detalles superfluos que podrían ocultar los conceptos e impedir la comprensión total del material.

	\item \textbf{Fórmulas y términos clave encerrados en recuadro}
	
Las fórmulas importantes se encierran en un recuadro como una forma de ayudar a los estudiantes a clasificar qué es esencial y qué no; asimismo, se definen y destacan términos clave, a fin de asegurar que los estudiantes perciban claramente la esencia de la materia.
\end{itemize}
